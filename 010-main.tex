
\chapter[Working title: Calculations]
{Working title: Calculations\label{ch1}}


\chapauth{Martin R. Hediger$^{a}$, Harm Otten$^{b}$
\chapaff{$^{a}$Z\"urich\\
$^{b}$K{\o}benhavn\\
ma.hed@bluewin.ch\\
harm82@gmail.com}}


\section{Motivation}\label{sec:mot}

Let us imagine two companies A and B.
Both companies use very similar technical equipment to carry out a biotechnological process where a chemical reaction is catalyzed by an enzyme.
Company A uses an enzyme with a rate constant $k_\text{A} = 1000s^{-1}$ while company B uses an enzyme with $k_\text{B} = 2000s^{-1}$.
Letting all other things be equal, the process of company B will therefore only require half the time to produce one Mole of product compared to the time required for company A.
Company B therefore can save energy required to heat up the reaction volume and the commercial implications of this are immediate.
The need for efficient catalysts\footnote{We use the terms \textit{enzme} and \textit{bio-}/\textit{catalyst} interchangeably.} arises from such an outline.
Increasing the performance of enzymes however is still far from trivial and forms a growing body of research.
What is clear though is that the development of such catalysts is costly, in terms of manpower, material and energy -- if it is carried out in the laboratory.
These costs can be saved if the development is carried out \textit{in silico}.
A number of companies have in fact formed around this quest: Novozymes (DK), Genzyme (US) or DSM (NL) to name but a few\citep{meyer2013use}.






\section{Applications}\label{sec:apps}

\section{Outlook}\label{sec:out}
